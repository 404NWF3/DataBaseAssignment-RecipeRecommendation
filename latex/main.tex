\documentclass[a4paper,12pt]{ctexart}

% --- 宏包引入 ---
\usepackage{geometry}       % 页面边距设置
\usepackage{graphicx}       % 图片支持
\usepackage{float}          % 图片/表格浮动控制
\usepackage{booktabs}       % 三线表支持 (toprule, midrule, bottomrule)
\usepackage{tabularx}       % 自动宽度的表格
\usepackage{longtable}      % 跨页长表格
\usepackage{listings}       % 代码高亮
\usepackage{xcolor}         % 颜色支持
\usepackage{fancyhdr}       % 页眉页脚
\usepackage{titlesec}       % 标题格式自定义
\usepackage{hyperref}       % 超链接和书签

% --- 页面设置 ---
\geometry{left=2.5cm, right=2.5cm, top=2.5cm, bottom=2.5cm}
\pagestyle{fancy}
\fancyhf{}
\fancyhead[C]{食谱推荐网站数据库方案设计与实现}
\fancyfoot[C]{\thepage}

% --- 代码高亮设置 (SQL) ---
\definecolor{codegreen}{rgb}{0,0.6,0}
\definecolor{codegray}{rgb}{0.5,0.5,0.5}
\definecolor{codepurple}{rgb}{0.58,0,0.82}
\definecolor{backcolour}{rgb}{0.95,0.95,0.92}

\lstset{
    language=SQL,
    backgroundcolor=\color{backcolour},   
    commentstyle=\color{codegreen},
    keywordstyle=\color{magenta},
    numberstyle=\tiny\color{codegray},
    stringstyle=\color{codepurple},
    basicstyle=\ttfamily\small,
    breakatwhitespace=false,         
    breaklines=true,                 
    captionpos=b,                    
    keepspaces=true,                 
    numbers=left,                    
    numbersep=5pt,                  
    showspaces=false,                
    showstringspaces=false,
    showtabs=false,                  
    tabsize=2,
    frame=single
}

% --- 自定义命令 ---
% 用于快速生成数据库表结构的命令
\newcommand{\dbtableheader}{
    \toprule
    \textbf{字段名} & \textbf{数据类型} & \textbf{非空} & \textbf{主键} & \textbf{说明} \\
    \midrule
    \endhead
    \bottomrule
    \endfoot
}

\begin{document}

% ==================== 封面部分 (还原封面.doc) ====================
\begin{titlepage}
    \centering
    \vspace*{1cm}
    
    {\Huge \textbf{综合实验报告}} \\
    \vspace{2cm}
    
    \renewcommand{\arraystretch}{1.5}
    \begin{tabular}{rl}
        \textbf{\large 实~验~名~称:} & \underline{\makebox[9cm]{食谱推荐网站数据库方案设计与实现}} \\
        \textbf{\large 实验项目类型:} & \underline{\makebox[9cm]{综合实验—学期项目}} \\
        \textbf{\large 实~~验~~室:} & \underline{\makebox[9cm]{云计算研究中心实验室520}} \\
        \textbf{\large 所属课程名称:} & \underline{\makebox[9cm]{《数据库》}} \\
        \textbf{\large 实~验~日~期:} & \underline{\makebox[9cm]{2025.9 ---- 2025.12}} \\
    \end{tabular}
    
    \vspace{2cm}
    
    \begin{tabular}{rl}
        \textbf{\large 小组成员(签名):} & \underline{\makebox[9cm]{(在此填入组员姓名)}} \\
        \textbf{\large 课程序号:} & \underline{\makebox[9cm]{}} \\
        \textbf{\large 成~~~~~绩:} & \underline{\makebox[9cm]{}} \\
    \end{tabular}

    \vfill
    {\large \textbf{【实验环境】} Oracle 11g 或更高版本} \\
    \vspace{1cm}
    \today
\end{titlepage}

% ==================== 目录 ====================
\tableofcontents
\newpage

% ==================== 正文部分 ====================

\section{设计部分}

\subsection{项目背景介绍}
AllRecipes是全球最大的食谱分享平台之一,拥有数百万用户和数百万条食谱。该网站以其用户友好的界面、丰富的食谱内容和强大的社区功能而闻名。

核心业务数据包括:
\begin{itemize}
    \item \textbf{用户(Users):} 平台使用者
    \item \textbf{食谱(Recipes):} 烹饪菜谱
    \item \textbf{食材(Ingredients):} 烹饪所需原料
    \item \dots (此处省略其他业务对象,请自行补充)
\end{itemize}

\subsection{ER图设计(概念模型)}
% 插入ER图的示例代码,请确保 figures 文件夹下有图片
\begin{figure}[H]
    \centering
    % \includegraphics[width=1.0\textwidth]{figures/er_diagram.png} 
    \fbox{\parbox{10cm}{\centering [此处插入 ER 图] \\ 请替换代码中的图片路径}}
    \caption{AllRecipes 数据库概念模型 (ER图)}
    \label{fig:er_diagram}
\end{figure}

\subsection{数据库库表设计(逻辑模型)}
本项目严格遵循 BCNF 规范。

\subsection{表的详细设计(物理模型)}
本系统共包含 26 张表。以下是核心表的详细设计:

% 示例:使用 longtable 处理长表格,即使跨页也能显示表头
% 这里展示 USERS 表的迁移效果
\begin{longtable}{p{3cm} p{3cm} c c p{4cm}}
    \caption{USERS 表(用户表)} \label{tab:users} \\
    \dbtableheader
    
    USER\_ID & NUMBER(10) & $\checkmark$ & $\checkmark$ & 用户唯一标识,自增 \\
    USERNAME & VARCHAR2(50) & $\checkmark$ & & 用户名,唯一 \\
    EMAIL & VARCHAR2(100) & $\checkmark$ & & 邮箱,唯一 \\
    PASSWORD\_HASH & VARCHAR2(255) & $\checkmark$ & & 加密后的密码哈希值 \\
    ACCOUNT\_STATUS & VARCHAR2(20) & $\checkmark$ & & 状态(active/suspended) \\
    CREATED\_AT & TIMESTAMP & $\checkmark$ & & 创建时间 (Default: SYSDATE) \\
    
\end{longtable}

\section{实现部分}

\subsection{表格创建}
以下是部分核心表的创建语句,包含完整性约束:

\begin{lstlisting}[caption={用户表创建脚本}]
CREATE TABLE USERS (
    user_id NUMBER(10) PRIMARY KEY,
    username VARCHAR2(50) NOT NULL UNIQUE,
    email VARCHAR2(100) NOT NULL UNIQUE,
    password_hash VARCHAR2(255) NOT NULL,
    account_status VARCHAR2(20) DEFAULT 'active' 
        CHECK (account_status IN ('active', 'inactive', 'suspended')),
    created_at TIMESTAMP DEFAULT SYSDATE
);
\end{lstlisting}

\subsection{视图设计方案}
系统设计了多张视图以支持业务查询,例如 \texttt{RECIPE\_DETAIL} 视图。

\subsection{数据库安全的实现}
\begin{itemize}
    \item \textbf{应用用户 (APP\_USER)}: 拥有业务数据的增删改查权限。
    \item \textbf{管理员 (DBA\_ADMIN)}: 拥有 DBA 最高权限。
\end{itemize}

\begin{lstlisting}[caption={创建用户及赋权}]
-- 创建应用用户
CREATE USER app_user IDENTIFIED BY "SecureP@ss123";
GRANT CONNECT, RESOURCE TO app_user;

-- 限制某些用户无法看到密码哈希
GRANT SELECT(user_id, username, email) ON USERS TO report_user;
\end{lstlisting}

\section{总结}
\subsection{组员分工}
\subsection{难点与不足分析}
\subsection{收获或体会及建议}

\end{document}